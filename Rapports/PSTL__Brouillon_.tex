\documentclass{article}

\usepackage[french]{babel}
\usepackage[T1]{fontenc}

\usepackage{multirow}
\usepackage{listings}
\usepackage{xcolor}
\definecolor{LightGray}{gray}{0.9}
\usepackage{minted}
\usemintedstyle{perldoc}


\colorlet{punct}{red!60!black}
\definecolor{background}{HTML}{EEEEEE}
\definecolor{delim}{RGB}{20,105,176}
\colorlet{numb}{magenta!60!black}

\lstdefinelanguage{json}{
    basicstyle=\normalfont\ttfamily,
    numbers=left,
    numberstyle=\scriptsize,
    stepnumber=1,
    numbersep=8pt,
    showstringspaces=false,
    breaklines=true,
    frame=lines,
    backgroundcolor=\color{background},
    literate=
     *{0}{{{\color{numb}0}}}{1}
      {1}{{{\color{numb}1}}}{1}
      {2}{{{\color{numb}2}}}{1}
      {3}{{{\color{numb}3}}}{1}
      {4}{{{\color{numb}4}}}{1}
      {5}{{{\color{numb}5}}}{1}
      {6}{{{\color{numb}6}}}{1}
      {7}{{{\color{numb}7}}}{1}
      {8}{{{\color{numb}8}}}{1}
      {9}{{{\color{numb}9}}}{1}
      {:}{{{\color{punct}{:}}}}{1}
      {,}{{{\color{punct}{,}}}}{1}
      {\{}{{{\color{delim}{\{}}}}{1}
      {\}}{{{\color{delim}{\}}}}}{1}
      {[}{{{\color{delim}{[}}}}{1}
      {]}{{{\color{delim}{]}}}}{1},
}
\usepackage[T1]{fontenc}
\usepackage[french]{babel}
\usepackage{amsmath}
\usepackage{graphicx}
\usepackage{listings}
\usepackage[indent=15pt]{parskip}
\usepackage[letterpaper,top=2cm,bottom=2cm,left=3cm,right=3cm,marginparwidth=1.75cm]{geometry}
\usepackage[colorlinks=true, allcolors=blue]{hyperref}
\usepackage{courier} %% Sets font for listing as Courier.
\usepackage{listings, xcolor}
%%\DeclareUnicodeCharacter{2212}{-}
\lstset{
tabsize = 4, %% set tab space width
showstringspaces = false, %% prevent space marking in strings, string is defined as the text that is generally printed directly to the console
numbers = left, %% display line numbers on the left
commentstyle = \color{green}, %% set comment color
keywordstyle = \color{blue}, %% set keyword color
stringstyle = \color{red}, %% set string color
rulecolor = \color{black}, %% set frame color to avoid being affected by text color
basicstyle = \small \ttfamily , %% set listing font and size
breaklines = true, %% enable line breaking
numberstyle = \tiny,
}
\begin{document}
\begin{titlepage}
\begin{figure}[!htb]
    \centering
    \includegraphics[width=5cm]{uninsubria-logo.png}
\end{figure}

\begin{center}
    \Large{\textbf{Sorbonne Université}}
    \vspace{3mm}
    \\ \normalsize{Faculté des Sciences et Ingénierie}
    \vspace{6mm}
    \\ \normalsize{Master Informatique}
    \\ \normalsize{Parcours Science et Technologie du Logiciel}
    \\  \normalsize{(STL)}
    \vspace{13mm}
\end{center}

\vspace{8mm}
\begin{center}
    \LARGE{\textbf{Benchmarking de solutions optimistes pour génération de données test à partir de JSON Schema}}
\end{center}
\vspace{30mm}

\begin{minipage}[t]{0.47\textwidth}
	{\normalsize{\textbf{Auteurs}}{\normalsize\vspace{1mm}
    \\ \normalsize{Abdelkader Boumessaoud\\ Zaky Abdellaoui}}} \\
    
    {\normalsize{\textbf{Encadrants}}{\normalsize\vspace{1mm}
    \\ \normalsize{Mohamed-Amine Baazizi\\ Lyes Attouche}}}
\end{minipage}
\end{titlepage}
\newcommand{\json}[0]{JSON Schema}
\newcommand{\jsonsch}[0]{JSON Schema}
\newcommand{\jschon}[0]{\texttt{jschon}}
\newcommand{\jg}[0]{\texttt{json-data-generator}}
\newcommand{\jsf}[0]{\texttt{json-schema-faker}}
\newcommand{\je}[0]{\texttt{json-everything}}
\newpage
\tableofcontents
\newpage
\newcommand{\todo}[1]{\textcolor{blue}{\emph{[#1]}}}

\input{body}

\newpage

\begin{thebibliography}{8}

\bibitem{js}\json \hspace{1mm} https://json-schema.org
\bibitem{spec}\json \hspace{1mm} Documentation https://json-schema.org/understanding-json-schema/
\bibitem{jsf}\jsf \hspace{1mm} Générateur (Avril 2023) https://github.com/json-schema-faker/json-schema-faker
\bibitem{je}\je \hspace{1mm} Générateur (Avril 2023) https://github.com/gregsdennis/json-everything
\bibitem{jg}\jg \hspace{1mm} Générateur https://github.com/jimblackler/jsongenerator
\bibitem{dg}Lyes Attouche, Mohamed-Amine Baazizi, Dario Colazzo, Giorgio Ghelli, Carlo Sartiani, Stefanie Scherzinger \emph{Witness Generation for JSON Schema} https://arxiv.org/pdf/2202.12849.pdf
\bibitem{jschon}\jschon \hspace{1mm} Validateur (Avril 2023) https://jschon.dev/
\bibitem{ts}\json \hspace{1mm} Test Suite (Février 2023) https://github.com/json-schema-org/JSON-Schema-Test-Suite
\bibitem{ds1} Snowplow Dataset (Fevrier 2023) https://github.com/snowplow/iglu-central
\bibitem{ds2} WashingtonPost Dataset (Fevrier 2023) https://github.com/washingtonpost/ans-schema
\bibitem{ds3} Kubernetes Dataset (Fevrier 2023) https://github.com/instrumenta/kubernetes-json-schema
\bibitem{ds4} GitHub Dataset (Fevrier 2023) https://github.com/sdbs-uni-p/json-schema-corpus
\bibitem{}Lyes Attouche, Mohamed Amine Baazizi, Dario Colazzo, Francesco Falleni, Giorgio Ghelli, Cristiano Landi, Carlo Sartiani, Stefanie Scherzinger,  \emph{A Tool for JSON Schema Witness Generation}, EDBT 2021: 694-697
\bibitem{}Mohamed Amine Baazizi, Dario Colazzo, Giorgio Ghelli, Carlo Sartiani, Stefanie Scherzinger, \emph{Not Elimination and Witness Generation for JSON Schema}, CoRR abs/2104.14828 (2021)
\bibitem{}Pierre Bourhis, Juan L. Reutter, Fernando Suárez, Domagoj Vrgoc,  \emph{JSON: Data model, Query languages and Schema specification}, PODS’1
\bibitem{}A. Antonio, "Exploitez des données au format JSON", OpenClassrooms. 
https://openclassrooms.com/fr/courses/7697016-creez-des-pages-web-dynamiques-avec-javascript/7911021-
exploitez-des-donnees-au-format-json.

\end{thebibliography}

\end{document}